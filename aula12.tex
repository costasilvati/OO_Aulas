\documentclass[11pt,aspectratio=43,ignorenonframetext,t]{beamer}

% Presentation settings
\mode<presentation>{
  \usetheme[framenumber,titleframestart=1]{UoM_alex}
  \usefonttheme{professionalfonts} % using non standard fonts for beamer
  \usefonttheme{serif}             % set font to Arial
  \usepackage{fontspec}
  \setmainfont[Ligatures=TeX]{Arial}
}

% Handout settings
\mode<article>{
  \usepackage{fullpage}                  % use full page
  \usepackage{fontspec}                  % set font to Arial
    \setmainfont[Ligatures=TeX]{Arial}
  \setlength{\parskip}{1.5\baselineskip} % correct beamer line spacings
  \setlength{\parindent}{0cm}
  \usepackage{enumitem}
    \setlist[itemize]{topsep=0pt}
  \definecolor{uomlinkblue}{HTML}{0071BC}
}


% Packages

% Configurando layout para mostrar codigos C++
\usepackage{listings}
\lstset{
  language=HTML,
  basicstyle=\ttfamily\small, 
  keywordstyle=\color{blue}, 
  stringstyle=\color{red}, 
  commentstyle=\color{red}, 
  extendedchars=true, 
  showspaces=false, 
  showstringspaces=false, 
  numbers=left,
  numberstyle=\tiny,
  breaklines=true, 
  backgroundcolor=\color{green!10},
  breakautoindent=true, 
  captionpos=b,
  xleftmargin=0pt,
}

\usepackage{graphicx}  % for graphics files
\usepackage{amsmath}   % assumes amsmath package installed
  \allowdisplaybreaks[1] % allow eqnarrays to break across pages
\usepackage{amssymb}   % assumes amsmath package installed 
\usepackage{hyperref} % add hyperlinks to document. Settings are for accessiblity
  \hypersetup{
    colorlinks=true,
    linkcolor=uomlinkblue,
    filecolor=uomlinkblue,      
    urlcolor=uomlinkblue,
	pdflang={en-GB},
}
\usepackage[document]{ragged2e} % left aligned text for accessibility
% experimental - does fundamentally work, if with quite a bit of effort
%\usepackage{axessibility} % LaTeX readable equations for accessibility
%  \tagpdfsetup{tabsorder=structure,uncompress,activate-all,interwordspace=true}
%  \pdfextension catalog{/Lang (en-GB)}
%  \RequirePackage{luacode}
%  \directlua{require("axessibility.lua")}
\usepackage{unicode-math} % unicode maths for accessibility
\usepackage{pdfcomment} % for alt text for accessibility
\usepackage{rotating}  % allow portrait figures and tables
\usepackage{subfigure} % allow matrices of figures
\usepackage{float}     % allows H option on floats to force here placement
\usepackage{multirow}  % allows merging of rows in tables
\usepackage{tabularx}  % allows fixed width tables
\usepackage{ctable}    % modifies \hline for use in table
\usepackage{bm}        % allow bold fonts in equations
\usepackage{pgf}       % allow graphics manipulation
\usepackage{media9}    % allow interactive flash files to be embedded
  \addmediapath{../media}
\usepackage{etoolbox}
  \makeatletter \preto{\@verbatim}{\topsep=0pt \partopsep=0pt} \makeatother  
  
% Custom commands
\newcommand{\matlab}{\emph{\sc{Matlab}}}
\newcommand{\maple}{\emph{\sc{Maple}}}
\newcommand{\simulink}{\emph{\sc{Simulink}}}
\newcommand{\dc}{d.c.}
\newcommand{\ac}{a.c.}
\newcommand{\rms}{RMS}
\newcommand{\wgn}{{\tt wgn}}
\newcommand{\sus}[1]{$^{\mbox{\scriptsize #1}}$}
\newcommand{\sub}[1]{$_{\mbox{\scriptsize #1}}$}
\newcommand{\chap}[1]{Chapter~\ref{#1}}
\newcommand{\sect}[1]{Section~\ref{#1}}
\newcommand{\fig}[1]{Fig.~\ref{#1}}
\newcommand{\tab}[1]{Table~\ref{#1}}
\newcommand{\equ}[1]{(\ref{#1})}
\newcommand{\appx}[1]{Appendix~\ref{#1}}
\newcommand{\degree}{\ensuremath{^\circ}}
\newcommand{\Vrms}{Vrms}
\newcommand{\Vpp}{V\sub{pp}}
\newcommand{\otoprule}{\midrule[\heavyrulewidth]}         
\newcolumntype{Z}{>{\centering\arraybackslash}X}  % tabularx centered columns 
\makeatletter \DeclareRobustCommand{\em}{\@nomath\em \if b\expandafter\@car\f@series\@nil \normalfont \else \bfseries \fi} \makeatother
\newcounter{example_number} % keep track of the example questions



%%%%%%%%%%%%%%%%%% FRONT MATTER %%%%%%%%%%%%%%%%%%
\title{Análise Orientada a objetos}
\subtitle{Aula 11}
\author{Prof. Me. Juliana Costa-Silva}

\begin{document}

\maketitle
%%%%%%%%%%%%%%%%%% TITLE SLIDE %%%%%%%%%%%%%%%%%%
\mode<presentation>{ \frame{\titlepage \label{slide:a}}} 
%\begin{figure}[!ht] 
%\fbox{\includeslide[width=\textwidth]{slide:a}} \end{figure}

%------------------------------------------------------------------------
\mode<presentation>{
\begin{frame}
\frametitle{Na aula de hoje...} 
\tableofcontents 
\end{frame}
}

\section{Introdução}
\begin{frame}{Na última aula...}
 \begin{itemize}
  \item Introdução a CRUD;
  \item Construção de CRUD com Array;
  \item Leitura de arquivos texto;
 \end{itemize}
\end{frame}
%----------------------------------------------------------------------------
\section{Criação de Arquivos}
\begin{frame}{Criação de arquivos}

\lstinputlisting[linerange={62-64}]{
./codigo/CudCliente2/src/Arquivos.java }
\vspace{0.3cm}
- O método CriarArquivo, retorna um objeto do tipo \textbf{File}\\ 
\pause
- A classe \textit{JFileChooser} é responsável por exibir uma janela de 
navegação, o método \textit{setFileSelectionMode} permite delimitar que tipo de 
arquivo a janela irá selecionar. \\
\pause
- Para o código acima, utilizamos a opção 
\textcolor{green!99}{FILES\_ONLY}, que permite apenas a seleção de arquivos.\\

\end{frame}
%--------------------------------------------------------------------------
\begin{frame}{Testando o método CriarArquivo}{Classe principal}

Na sua classe principal:
\lstinputlisting[linerange={8-11}]{
./codigo/CudCliente2/src/CudCliente2.java}
% \vspace{0.3cm}
\pause 
\textcolor{red}{O que é guardado na variável result?} Execute a main e faça o 
teste!

\end{frame}
%----------------------------------------------------------------------------
\begin{frame}{CriarArquivo}{JFileChooser}
  
    \lstinputlisting[linerange={65-66}]{./codigo/CudCliente2/src/Arquivos.java }
\vspace{0.5cm}
\pause
  \begin{itemize}
    \item O método \textit{fileChooser.showSaveDialog(null);}, retorna:
    \item 1 se a operação for cancelada.
    \item 0 se o arquivo foi criado ou selecionado.
  \end{itemize}
 
\end{frame}
%----------------------------------------------------------------------------
\begin{frame}{CriarArquivo}{JFileChooser}
  Nosso método ficou assim:
\lstinputlisting[linerange={62-70}]{
./codigo/CudCliente2/src/Arquivos.java }

\pause
  \begin{itemize}
    \item Na linha 7 validamos se o usuário cancelou a operação;
    \pause \item O que devemos fazer agora?
  \end{itemize}
\end{frame}
%----------------------------------------------------------------------------
\begin{frame}{Método CriarArquivo}{Acrescentado...}
Acrescente mais algumas linhas ao método CriarArquivo
%     \lstinputlisting[linerange={60-63}]{
% ./codigo/CudCliente2/src/Arquivos.java }
\includegraphics[height=0.14\paperheight]{criarArquivo.png} \\
  \pause
  \begin{itemize}
   \item A variável diretorio recebe a ``pasta'' onde o arquivo foi salvo;
   \pause \item O método mkdir() cria a pasta caso ela não exista;
   \pause \item A string nomeArq recebe o nome do arquivo criado (o existente);
   \pause \item A variável arquivo do tipo File instancia um objeto arquivo, 
com  caminho no diretorio selecionado e com o nome definido pelo usuário.
  \end{itemize}
  \pause \textcolor{red}{Execute a main().}
\end{frame}
%----------------------------------------------------------------------------
\begin{frame}{Método CriarArquivo}{Acrescentado...}
Acrescente mais[2] algumas linhas ao método CriarArquivo
    \lstinputlisting[linerange={75-82}]{
./codigo/CudCliente2/src/Arquivos.java }
% \includegraphics[height=0.14\paperheight]{criarArquivo.png} \\
  \pause
  \begin{itemize}
   \item O processo para a criação de um escritor é igual ao de criação de 
leitor;
  \end{itemize}
  \pause \textcolor{red}{Execute a main().}
\end{frame}

%----------------------------------------------------------------
\section{Escrita em arquivos}
\begin{frame}{Método Escreve}{BufferedWriter}
Na classe de manipulação de arquivos crie o método escreve:
    \lstinputlisting[linerange={85-94}]{
./codigo/CudCliente2/src/Arquivos.java }
% \includegraphics[height=0.14\paperheight]{criarArquivo.png} \\
  \pause
  \begin{itemize}
   \item O método recebe um BufferedWriter (escritor), uma String (texto que 
será escrito) e um valor booleano (indica se o buffer deve ser descarregado ou 
não);
  \end{itemize}

\end{frame}
%----------------------------------------------------------------------
\section{Atividade I}
\begin{frame}{Atividade I}{\textcolor{red}{Biblioteca Virtual}}
 Busque o livro abaixo na Biblioteca Virtual:
  \begin{columns}
   \begin{column}{0.4\textwidth}
     Java: Como programar\\
     Capítulo 7: \cite{deitel2017java}\\
     Páginas: 225 a 227\\
     
     \begin{enumerate}
      \item Leia e Implemente a seção 7.16 \textbf{coleções e a classe 
ArrayList};
     \end{enumerate}     
   \end{column}
   \begin{column}{0.3\textwidth}
    \begin{center}
  \includegraphics[height=0.5\paperheight]{deitel2017java.png} \\
 \end{center}
   \end{column}
 \end{columns}
 
 \end{frame}
 %----------------------------------------------------------------------------
\section{Entrada de Dados}
\mode<presentation>{\begin{frame}{Entrada de dados}
  A leitura do console (teclado) obtém dados a partir do objeto   
\textit{System.in}.
  \begin{itemize}
    \item Objetos System.in leem somente bytes;
    \item Por isso é necessário transforma-lo em um objeto 
   \textit{InputStreamReader}.
 \item O \textit{InputStreamReader} lê somente caracteres, então criamos 
um \textit{BufferedReader}.
   \item O \textit{BufferedReader} é um objeto que acopla vários 
caracteres lidos (uma frase por exemplo).
  \end{itemize}
\end{frame}}
%----------------------------------------------------------------------------
\mode<presentation>{\begin{frame}{Utilizando \textit{BufferedReader}}
   Criando um objeto \textit{BufferedReader}
\small{\lstinputlisting[linerange={9-11}]{
./cod/Aula7Leitura.java } }
Declaração da String que receberá a leitura:
\small{\lstinputlisting[linerange={12-13}]{
./cod/Aula7Leitura.java } }\
\end{frame}}
%----------------------------------------------------------------------------
\mode<presentation>{\begin{frame}{Utilizando \textit{BufferedReader}}
  Executando a leitura:
\small{\lstinputlisting[linerange={14-20}]{
./cod/Aula7Leitura.java } }
Para saber sobre as diferenças da classe Scanner leia:\\
\href{https://www.devmedia.com.br/entrada-de-dados-classe-scanner/21366}[
https://www.devmedia.com.br/entrada-de-dados-classe-scanner/21366]
\end{frame}}
 %----------------------------------------------------------------------
\begin{frame}{Atividade II}
 \begin{enumerate}
   \setcounter{enumi}{1}
  \item Atualize a atividade do CRUD de cliente, utilizando leitura e escrita 
em arquivo;
  \item Utilize também ArrayList no lugar dos Arrays convencionais;
 \end{enumerate}
 \end{frame}
%-----------------------------------------------------------------------
\section{Leitura recomendada}
\begin{frame}{Leitura complementar}
 Para mais informações sobre arrays e ArrayList em JAVA, leia:\\
 \begin{columns}
   \begin{column}{0.4\textwidth}
     Java: Como programar 10{\textordfeminine} edição\\
     Capítulo 7: \cite{deitel2017java}\\
   \end{column}
   \begin{column}{0.3\textwidth}
    \begin{center}
  \includegraphics[height=0.5\paperheight]{deitel2017java.png} \\
 \end{center}
   \end{column}
 \end{columns}
\end{frame}
\end{document}