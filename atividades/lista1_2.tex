%%% Template para anotações de aula
%%% Feito por Daniel Campos com base no template de Willian Chamma que fez com base no template de Mikhail Klassen



\documentclass[12pt,a4paper, brazil]{article}

%%%%%%% INFORMAÇÕES DO CABEÇALHO
\newcommand{\workingDate}{\textsc{\selectlanguage{portuguese}\today}}
\newcommand{\userName}{DesSoftware}
\newcommand{\institution}{Universidade Positivo - Londrina}
\usepackage{researchdiary_png}




\begin{document}

\begin{center}
{\textbf {\huge Atividade Prática 1.2}}\\[5mm]
{\large Desenvolvimento de Software 2023-1} \\
{\large Prof. Me. Juliana Costa Silva} \\
\today\\[5mm] %% se quiser colocar data
\end{center}
% TEMPLATE DE ATIVIDADE: https://www.overleaf.com/latex/templates/template-para-relatorio-tecnico-memorial-descritivo-utfpr/ywygvmqcqysy

\section*{Orientações}

\begin{itemize}
  \item Desenvolva todas as atividades em linguagem Java, outras linguagens não serão aceitas na entrega;
  \item É ideal compreender o código desenvolvido, entende o problema e, planeje a solução antes de implementar;
  \item \textbf{Esta lista terá uma das atividades sorteada para resolução em sala de aula, manuscrita (no papel)}. Esta atividade será realizada unicamente no dia 03\/04\/2023 as 19:00hs com 20 minutos para entrega e sem consulta;
  \item Será avaliada apenas a atividade entregue no papel, com enfase na lógica de programação e uso correto dos conceitos de Orientação a Objetos, \textit{pequenos erros de sintaxe} serão desconsiderados;
  \item A lista de atividades deve ser resolvida individualmente. Os alunos podem discutir as soluções e eventualmente ajudar a resolução de um colega, porém a cópia, integral ou parcial de código não é permitida. Desse modo alunos com códigos iguais ou com alta similaridade terão suas atividades ANULADAS, recebendo 0,0 como nota nesta atividade.
\end{itemize}


\vspace{0.5cm}


\section{Elementos de Construção}
\par
Desenvolva uma aplicação que trabalhe com a classe Door \\
\textbf{Atributos:} open, color, width, height, depth;\\
\textbf{Métodos:} void open(), void close(),void paint(String s), boolean isOpen().\\

\par
Na classe que contém o método main: Crie uma Door, open e close a mesma, paint de diversas cores, altere suas dimensões e use o método isOpen para verificar se ela está aberta.

 \dotfill

\section{Construindo Objetos Compostos}
\par
Desenvolva uma aplicação que trabalhe com a classe House;\\
\textbf{Atributos:} color, door1, door2, door3 (Use a Classe Door da questão anterior);\\
\textbf{Método:} void paint(String s), int qhowDoorIsOpen();\\

Na Classe que contém o método main: Crie uma casa e pinte-a. Crie três portas e coloque-as na casa; abra e feche as mesmas como desejar. Utilize o métodoquantasPortasEstaoAbertaspara imprimir o número de portas abertas.

\dotfill

\section{Datas}

\par 
Crie uma classe para representar datas.\\
\begin{itemize}
    \item Apresente a data utilizando três atributos: dia, mês, e ano.\\
    \item Sua classe deve ter um construtor que inicializa os três atributos e verifica a validade dos valores fornecidos.
    \item Desenvolva um construtor sem parâmetros que inicializa a data com a data atual fornecida pelo sistema operacional.
    \item Desenvolva um método set um get para cada atributo.
    \item Desenvolva o método toString para retornar uma representação de data como string. Considere que a data deve ser formatada mostrando o dia, o mês e o ano separados por barra (/).
    \item Desenvolva um método para avançar uma data para o dia seguinte.
    \item Escreva um aplicativo de teste que demonstra as capacidades da classe.
\end{itemize}

\dotfill

\section{Números}
\par
Crie uma classe ConjuntoInteiro para representar um conjunto de números inteiros.
\\
\begin{itemize}
    \item Cada objeto da classe ConjuntoInteiro pode armazenar inteiros no intervalo de 0 até um valor máximo específico para cada objeto. 
    \item O conjunto deve ser representado por um array de booleanos. O elemento do array na posição i é verdadeiro se e somente se o inteiro i pertencer ao conjunto.
    \item O construtor inicializa o objeto como um conjunto vazio (isto é, um conjunto cuja representação de array contém todos os valores falso).
    \item Desenvolva métodos para implementar as operações de união, interseção e diferença de conjuntos.
    \item Desenvolva um método para inserir um novo elemento no conjunto e outro método para excluir um elemento do conjunto.
    \item Desenvolva ainda um método para converter um conjunto para string. Faça uma aplicação para testar a classe.
\end{itemize}


\dotfill

\end{document}