\documentclass[11pt,aspectratio=43,ignorenonframetext,t]{beamer}

% Presentation settings
\mode<presentation>{
  \usetheme[framenumber,titleframestart=1]{UoM_alex}
  \usefonttheme{professionalfonts} % using non standard fonts for beamer
  \usefonttheme{serif}             % set font to Arial
  \usepackage{fontspec}
  \setmainfont[Ligatures=TeX]{Arial}
}

% Handout settings
\mode<article>{
  \usepackage{fullpage}                  % use full page
  \usepackage{fontspec}                  % set font to Arial
    \setmainfont[Ligatures=TeX]{Arial}
  \setlength{\parskip}{1.5\baselineskip} % correct beamer line spacings
  \setlength{\parindent}{0cm}
  \usepackage{enumitem}
    \setlist[itemize]{topsep=0pt}
  \definecolor{uomlinkblue}{HTML}{0071BC}
}


% Packages

% Configurando layout para mostrar codigos C++
\usepackage{listings}
\lstset{
  language=HTML,
  basicstyle=\ttfamily\small, 
  keywordstyle=\color{blue}, 
  stringstyle=\color{red}, 
  commentstyle=\color{red}, 
  extendedchars=true, 
  showspaces=false, 
  showstringspaces=false, 
  numbers=left,
  numberstyle=\tiny,
  breaklines=true, 
  backgroundcolor=\color{green!10},
  breakautoindent=true, 
  captionpos=b,
  xleftmargin=0pt,
}

\usepackage{graphicx}  % for graphics files
\usepackage{amsmath}   % assumes amsmath package installed
  \allowdisplaybreaks[1] % allow eqnarrays to break across pages
\usepackage{amssymb}   % assumes amsmath package installed 
\usepackage{hyperref} % add hyperlinks to document. Settings are for accessiblity
  \hypersetup{
    colorlinks=true,
    linkcolor=uomlinkblue,
    filecolor=uomlinkblue,      
    urlcolor=uomlinkblue,
	pdflang={en-GB},
}
\usepackage[document]{ragged2e} % left aligned text for accessibility
% experimental - does fundamentally work, if with quite a bit of effort
%\usepackage{axessibility} % LaTeX readable equations for accessibility
%  \tagpdfsetup{tabsorder=structure,uncompress,activate-all,interwordspace=true}
%  \pdfextension catalog{/Lang (en-GB)}
%  \RequirePackage{luacode}
%  \directlua{require("axessibility.lua")}
\usepackage{unicode-math} % unicode maths for accessibility
\usepackage{pdfcomment} % for alt text for accessibility
\usepackage{rotating}  % allow portrait figures and tables
\usepackage{subfigure} % allow matrices of figures
\usepackage{float}     % allows H option on floats to force here placement
\usepackage{multirow}  % allows merging of rows in tables
\usepackage{tabularx}  % allows fixed width tables
\usepackage{ctable}    % modifies \hline for use in table
\usepackage{bm}        % allow bold fonts in equations
\usepackage{pgf}       % allow graphics manipulation
\usepackage{media9}    % allow interactive flash files to be embedded
  \addmediapath{../media}
\usepackage{etoolbox}
  \makeatletter \preto{\@verbatim}{\topsep=0pt \partopsep=0pt} \makeatother  
  
% Custom commands
\newcommand{\matlab}{\emph{\sc{Matlab}}}
\newcommand{\maple}{\emph{\sc{Maple}}}
\newcommand{\simulink}{\emph{\sc{Simulink}}}
\newcommand{\dc}{d.c.}
\newcommand{\ac}{a.c.}
\newcommand{\rms}{RMS}
\newcommand{\wgn}{{\tt wgn}}
\newcommand{\sus}[1]{$^{\mbox{\scriptsize #1}}$}
\newcommand{\sub}[1]{$_{\mbox{\scriptsize #1}}$}
\newcommand{\chap}[1]{Chapter~\ref{#1}}
\newcommand{\sect}[1]{Section~\ref{#1}}
\newcommand{\fig}[1]{Fig.~\ref{#1}}
\newcommand{\tab}[1]{Table~\ref{#1}}
\newcommand{\equ}[1]{(\ref{#1})}
\newcommand{\appx}[1]{Appendix~\ref{#1}}
\newcommand{\degree}{\ensuremath{^\circ}}
\newcommand{\Vrms}{Vrms}
\newcommand{\Vpp}{V\sub{pp}}
\newcommand{\otoprule}{\midrule[\heavyrulewidth]}         
\newcolumntype{Z}{>{\centering\arraybackslash}X}  % tabularx centered columns 
\makeatletter \DeclareRobustCommand{\em}{\@nomath\em \if b\expandafter\@car\f@series\@nil \normalfont \else \bfseries \fi} \makeatother
\newcounter{example_number} % keep track of the example questions



%%%%%%%%%%%%%%%%%% FRONT MATTER %%%%%%%%%%%%%%%%%%
\title{Análise Orientada a objetos}
\subtitle{Aula 11}
\author{Prof. Me. Juliana Costa-Silva}

\begin{document}

\maketitle
%%%%%%%%%%%%%%%%%% TITLE SLIDE %%%%%%%%%%%%%%%%%%
\mode<presentation>{ \frame{\titlepage \label{slide:a}}} 
%\begin{figure}[!ht] 
%\fbox{\includeslide[width=\textwidth]{slide:a}} \end{figure}

%------------------------------------------------------------------------
\mode<presentation>{
\begin{frame}
\frametitle{Na aula de hoje...} 
\tableofcontents 
\end{frame}
}
%******----------------------------------------------------------------------------
\section{Descrição do Projeto}
\mode<presentation>{
\begin{frame}{Projeto/ Avaliação}
\small
\begin{enumerate}
    \item A aplicação deve iniciar apenas com login, o usuário deve criar uma nova senha caso erre 3 vezes. 
    \item A nova senha deve ser diferente das 3 últimas; 
    \item Desenvolva uma aplicação com no mínimo 5 classes coerentes em relação ao problema escolhido;
    \item Modele no mínimo 3 classes compostas (Ex.: Classe Pessoa, com o Atributo Endereco desenvolvida em aula);
    \item Todas as classes devem possuir atributos privados, métodos get e set e método toString();
    \item A aplicação deve ter no mínimo um "relatório" que envolva várias classes;
    \item A aplicação deve conter um menu, com opções de cadastro e consulta a dados cadastrados;
    \item Os dados devem ser registrados no banco de dados/ ou em ArrayList;
    
\end{enumerate}
\end{frame}
}
%---------------------------------------
\section{Descrição da avaliação}
\mode<presentation>{
\begin{frame}{Como serei Avaliado}
\begin{enumerate}
    \item O aluno deve enviar o código desenvolvido via formulário até a data da avaliação as 19:00h;
    \item O aluno deve preparar uma apresentação do sistema, de no mínimo 2 minutos;
    \item O aluno deve responder no mínimo 2 perguntas do professor durante a avaliação;
    \item A pontuação será calculada com base no código desenvolvido e entregue 30\% da nota;
    \item Modelagem das classes e uso dos conceitos de orientação a objetos, assim como relacionamentos entre classes 20\% da nota;
    \item Perguntas respondidas 50\% da nota;
\end{enumerate}
\textcolor{red}{A avaliação é individual! Plágios receberão nota 0}
\end{frame}
}
%******-----------------------------------------------------------------------
\section{Leitura recomendada}
 \mode<presentation>{
\begin{frame}{Leitura complementar}
 Para mais informações sobre JAVA, leia:\\
 \begin{columns}
   \begin{column}{0.4\textwidth}
     Java: Como programar\\
     \textbf{Capítulo 24}:\\ 
      \cite{deitel2010java}
   \end{column}
   \begin{column}{0.3\textwidth}
    \begin{center}
  \includegraphics[height=0.5\paperheight]{fig/aula1/deitel2017java.png} \\
 \end{center}
   \end{column}
 \end{columns}
\end{frame}
}

%----------------------------------------------------------------------------------------------------------------------
 
 \mode<presentation>{\begin{frame}{Referências}%[allowframebreaks]
 \small
 \begin{center}
 	\bibliographystyle{apalike}
	 \bibliography{ref_aula_progI}
 \end{center}
 \end{frame}}


%%%%%%%%%%%%%%%%%% END MATTER %%%%%%%%%%%%%%%%%%
\end{document}